% !TeX spellcheck = ru_RU
% !TeX encoding = UTF-8
\section{Передача радио сигнала <<за горизонт>>. Эффект Кабанова}
На сегодняшний день частотный ресурс распределяется Государственной комиссией по радиочастотам.
Государственная комиссия по радиочастотам (ГКРЧ) — межведомственный координационный орган, действующий при Министерстве связи и массовых коммуникаций Российской Федерации. ГКРЧ обладает всей полнотой полномочий в области регулирования радиочастотного спектра и отвечает за формирование государственной политики в области его распределения и использования. Помимо этого, комиссия готовит позицию Администрации связи Российской Федерации на все форумы Международного союза электросвязи для защиты интересов страны на международном уровне и международно-правовой защиты орбитально-частотного ресурса Российской Федерации. В рамках Комиссии проводятся исследования, по совершенствованию механизмов регулирования использования радиочастотного спектра, обеспечению электромагнитной совместимости радиоэлектронных средств, решению проблем внедрения на сетях связи России новых радиотехнологий.

Долгое время после того, как было изобретено рядио, считалось, что для целей связи наиболее приемлемы
длинные волны, так как они позволяют устанавливать связь на больших расстояниях, чем короткие.
Казалось, что короткие волны, в отличие от длинных, не в состоянии распространяться на значительные
расстояния за горизонт. Теперь весь мир. пронизывает радиосвязь на коротких волнах, хотя до 1947 г.
никто не мог представить себе, чтобы радиосигнал, посланный на коротких волнах, можно было принять
в том же месте, откуда он послан.

Профессор, доктор технических наук Н. И. Кабанов (Новосибирский электротехнический институт) открыл
ранее неизвестное явление дальнего коротковолнового рассеяния радиоволн отдельными элементами
поверхности Земли. Радиоволны, излучаемые радиопередающим устройством под некоторым углом к горизонту,
отражаются ионосферой и идут обратно, к Земле. Часть их энергии рассеивается неоднородностями земной
поверхности и распространяется в разные стороны. Рассеянные радиоволны вновь отражаются от ионосферы
и возвращаются на Землю, причем какая-то их доля попадает и в то место, где находится радиопередающее устройство.

В 1950 г. Государственная комиссия под председательством академика А. И. Берга рассмотрела полученные
Н. И. Кабановым данные и дала следующее заключение: <<Настоящей работой впервые экспериментально установлено
существование регулярных рассеянных отражений от Земли на коротких волнах, что имеет принципиальное
значение для исследований условий распространения коротких волн, в частности применительно к эксплуатации
магистральных линий и средств дальней радионавигации>>.

Оригинальные эксперименты, поставленные Н. И. Кабановым, позволили обнаружить, что рассеяние радиоволн
гористыми участками Земли происходит более интенсивно, чем морями, подтвердили, что по границам дальности
отражений можно судить о состоянии ионосферы.

Использование эффекта Кабанова для исследования ионосферы (метод возвратно-наклонного зондирования)
дает возможность определять условия распространения радиоволн в радиусе до 9-12 тыс. км, т. е.
почти над четвертью поверхности земного шара. Метод возвратно-наклонного зондирования позволяет
значительно повысить надежность радиосвязи. Он особенно ценен тем, что используется в весьма загруженном
диапазоне коротких волн, обеспечивающих дальнюю радиосвязь. В нашей стране этот метод был разработан
и вошел в практику на два года раньше, чем за границей.

Эффект Кабанова находит применение также в ионосферной радиолокации и в других областях радиосвязи.
На основе эффекта Н. И. Кабанов и С. Г. Евскжов разработали способ радиолокационного загоризонтного
обзора поверхности Земли через ионизированные следы метеоров.

За рубежом эффект Кабанова получил всеобщее признание. Например, в Великобритании на ионосферной станции
в Слоу ведутся наблюдения за прохождением радиоволн с использованием коротковолнового рассеянного
отражения от Земли в радиусе до 6 тыс. км. В США разработаны сверхдальние загоризонтные радиолокаторы,
основанные на эффекте Кабанова.
\newpage