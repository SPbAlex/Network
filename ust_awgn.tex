% !TeX spellcheck = ru_RU
% !TeX encoding = UTF-8
\subsection{АБГШ}

Аддитивный Белый Гауссовский Шум (АБГШ) является видом мешающего воздействия в канале связи или других процессах. Определяется данный вид шума как гауссовский случайный процесс $n(t )$ с нулевым средним и спектральной плотностью мощности $S_n( f ) = N_0 / 2$. АБГШ является наиболее распространённым видом шума, используемым для расчёта и моделирования систем связи. Термин «аддитивный» означает, что данный вид шума суммируется с исходным сигналом и статистически не зависим от сигнала.

Дисперсия АБГШ может быть вычислена как $\sigma^2 = \int_{-\infty}^{\infty}S_n(f)df$. Так как АБГШ существует во всей полосе частот $-\infty < f < \infty$, то $\sigma^2 = \infty$. В реальности такого не может существовать, т.к. бесконечно большой мощности не может быть. Шум не может существовать без сигнала, таким образом, ширина полосы частот шума зависит от ширины полосы частот исходного сигнала. 