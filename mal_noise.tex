% !TeX spellcheck = ru_RU
% !TeX encoding = UTF-8
\subsection{Отношение сигнал/шум}
Отношение сигнал/шум --- безразмерная величина, равная отношению мощности полезного сигнала к мощности шума.
\begin{equation}
SNR= \dfrac{P_signal}{P_noise}
\end{equation} 
где $P_signal$ --- средняя мощность сигнала,где $P_noise$ --- средняя мощность шума,
Основными параметрами системы передачи являются скорость передачи, ширина полосы частот и отношение сигнал/шум. Эти параметры обычно являются исходными, и при заданных значениях этих параметров требуется обеспечить требуемое качество передачи.
Величина
$(E/N_0)$ называется отношением сигнал/шум.  В инженерной практике принято выражать это отношение в децибелах(дБ).  Величина отношения, измеренная в дБ, определяется как
\begin{equation}
(E/N_0)_db=10log⁡(\dfrac{E}{N_{0}})
\end{equation} 
и наоборот
\begin{equation}
(E/N_0)=10^{(\dfrac{E}{N_{0}})_db/10}   
\end{equation} 