% !TeX spellcheck = ru_RU
% !TeX encoding = UTF-8
\section{Отношение сигнал/шум}
Отношение сигнал/шум --- безразмерная величина, равная отношению мощности полезного сигнала к мощности шума.

\[ SNR=P_{signal}/P_{noise} \]

где $P_{signal}$ --- средняя мощность сигнала, а $P_{noise}$ --- средняя мощность шума,
Основными параметрами системы передачи являются скорость передачи, ширина полосы частот и отношение сигнал/шум. Эти параметры обычно являются исходными, и при заданных значениях этих параметров требуется обеспечить требуемое качество передачи.
Величина $(E/N_0)$ называется отношением сигнал/шум.  В инженерной практике принято выражать это отношение в децибелах(дБ).  Величина отношения, измеренная в дБ, определяется как

\[(E/N_0)_{db}=10ln⁡(\dfrac{E}{N_{0}})\]
и наоборот

\[(E/N_0)=10^{(\dfrac{E}{N_{0}})_{db}/10} \]   
