% !TeX encoding = UTF-8
% !TeX spellcheck = ru_RU
\documentclass[a4paper,14pt]{extarticle} %размер бумаги устанавливаем А4, шрифт 14пунктов
\usepackage{amssymb,amsfonts,amsmath,mathtext,cite,enumerate,float} %подключаем нужные пакеты расширений
\usepackage[T2A, T1]{fontenc}
\usepackage{cmap}
\usepackage[utf8]{inputenc}%включаем свою кодировку: koi8-r или utf8 в UNIX, cp1251 в Windows
\usepackage[english, russian]{babel}%используем русский и английский языки с переносами

\usepackage{graphicx} %хотим вставлять в диплом рисунки?
\graphicspath{{images/}}%путь к рисункам

%{tikz}
\usepackage{pgfplots}

\usepackage{listings}
\lstset{
	frame=single,
	breaklines=true
}
\lstset{language=C++,
	basicstyle=\ttfamily,
	keywordstyle=\color{blue}\ttfamily,
	stringstyle=\color{red}\ttfamily,
	commentstyle=\color{green}\ttfamily,
	morecomment=[l][\color{magenta}]{\#}
}

\usepackage[colorlinks=false,pdfborder={0 0 0}]{hyperref} %использование гиперссылок, colorlinks - цвет текста ссылки, pdfborder - окантовка. 

%шрифт Times New Roman
%\usepackage{fontspec}
%\setmainfont{Times New Roman}
%\setallmainfonts{Times New Roman}

\usepackage{titlesec}
\titleformat{\section}[block]
{\filcenter\large}
{\thesection}
{1em}{\MakeUppercase}
\titlespacing*{\section}{0pt}{-30pt}{*4}

\makeatletter
%\renewcommand{\@biblabel}[1]{#1.} % Заменяем библиографию с квадратных скобок на точку:
\makeatother


\usepackage{geometry} % Меняем поля страницы
\geometry{left=3cm}% левое поле
\geometry{right=15mm}% правое поле
\geometry{top=2cm}% верхнее поле
\geometry{bottom=2cm}% нижнее поле
\linespread{1.5}

\usepackage{indentfirst} % отделять первую строку раздела абзацным отступом
\setlength\parindent{5ex}

%links
\usepackage{url}

\usepackage[tableposition=top,singlelinecheck=false, justification=centering]{caption}
\usepackage{subcaption}

%  маркированные списки
\renewcommand{\labelitemi}{--}
\renewcommand{\labelitemii}{--}
%  нумерованные списки
\renewcommand{\labelenumi}{\asbuk{enumi})}
\renewcommand{\labelenumii}{\arabic{enumii})}

% номер сноски со скобкой
\renewcommand*{\thefootnote}{\arabic{footnote})}
\renewcommand{\footnoterule}{%
	\kern -3pt
	\hrule width 40mm height .4pt
	\kern 2.6pt
}

%иллюстрации и таблицы
\DeclareCaptionLabelFormat{gostfigure}{Рисунок #2}
\DeclareCaptionLabelFormat{gosttable}{Таблица #2}
\DeclareCaptionLabelSeparator{gost}{~---~}
\captionsetup{labelsep=gost}
\captionsetup*[figure]{labelformat=gostfigure}
\captionsetup*[table]{labelformat=gosttable}
\renewcommand{\thesubfigure}{\asbuk{subfigure}}


%\renewcommand{\rmdefault}{ftm}
\renewcommand{\figurename}{Рисунок} % Рис -> Рисунок
%\usepackage[labelsep=period,labelfont=bf,figurename={Рисунок},figurewithin=none]{caption}
\renewcommand{\theenumi}{\arabic{enumi}}% Меняем везде перечисления на цифра.цифра
\renewcommand{\labelenumi}{\arabic{enumi}}% Меняем везде перечисления на цифра.цифра
\renewcommand{\theenumii}{.\arabic{enumii}}% Меняем везде перечисления на цифра.цифра
\renewcommand{\labelenumii}{\arabic{enumi}.\arabic{enumii}.}% Меняем везде перечисления на цифра.цифра
\renewcommand{\theenumiii}{.\arabic{enumiii}}% Меняем везде перечисления на цифра.цифра
\renewcommand{\labelenumiii}{\arabic{enumi}.\arabic{enumii}.\arabic{enumiii}.}% Меняем везде перечисления на цифра.цифра

\usepackage{tocloft}
\renewcommand{\cftsecleader}{\cftdotfill{\cftdotsep}}
%\renewcommand{\cfttoctitlefont}{\Large\filcenter}

%\setcounter{page}{2} %нумерация страниц с 3
%\addto\captionsrussian{\renewcommand\contentsname{СОДЕРЖАНИЕ}}
%\addto\captionsrussian{\renewcommand\refname{СПИСОК ИСПОЛЬЗОВАНЫХ ИСТОЧНИКОВ}}



\begin{document}
	% !TeX spellcheck = ru_RU
% !TeX encoding = UTF-8
\subsection{АБГШ}
\label{sec:awgn}

Аддитивный Белый Гауссовский Шум (АБГШ) является видом мешающего воздействия в канале связи или других процессах. Определяется данный вид шума как гауссовский случайный процесс $n(t )$ с нулевым средним и спектральной плотностью мощности $S_n( f ) = N_0 / 2$. АБГШ является наиболее распространённым видом шума, используемым для расчёта и моделирования систем связи. Термин «аддитивный» означает, что данный вид шума суммируется с исходным сигналом и статистически не зависим от сигнала.

Дисперсия АБГШ может быть вычислена как $\sigma^2 = \int_{-\infty}^{\infty}S_n(f)df$. Так как АБГШ существует во всей полосе частот $-\infty < f < \infty$, то $\sigma^2 = \infty$. В реальности такого не может существовать, т.к. бесконечно большой мощности не может быть. Шум не может существовать без сигнала, таким образом, ширина полосы частот шума зависит от ширины полосы частот исходного сигнала. 
	% !TeX spellcheck = ru_RU
% !TeX encoding = UTF-8
\section{Теоретическая вероятность в двоичном канале с аддитивным белым гуассовским шумом}

Рассмотрим систему передачи двоичных сигналов $0$ и $1$. Вероятность ошибки в такой системе определяется по следующей формуле:

\begin{equation}
	P_e = \sum_{i=0}^{1}P_e(i)P_i = P_e(0)P_0 + P_e(1)P_1,
\end{equation}

где $P_e(i)$ -- условная вероятность ошибки при передаче $i$-го сигнала, $P_i$ -- вероятность передачи $i$-го сигнала. Рассмотрим простую систему с одинаковыми вероятностями передачи сигналов. Таким образом, можно рассчитать вероятность ошибки для одного символа, эта же вероятность будет являться вероятностью ошибки для всей системы. Вероятность ошибки будем рассчитывать по максимальному правдоподобию:
\begin{equation}
	\label{form:pe_1}
	P_e(0) = Pr[d^2(r,s_0) > d^2(r,s_1) \mid 0] = Pr[ || r - s_0||^2 > ||r - s_1||^2 \mid 0] ,
\end{equation}

где $r$ -- принятый сигнал, $s_i$ -- переданный $i$-ый сигнал. При этом нужно учесть, что $r = s_i + n$, где $n$ -- АБГШ, который описан в разделе \ref{sec:awgn}. Тогда формула (\ref{form:pe_1}) приобретает вид:
\begin{equation}
	P_e(0) = Pr[||n||^2 > ||s_0 - s_1 + n||^2] = Pr[||n||^2 - ||s_0 - s_1 + n||^2 > 0].
\end{equation}
Выражение $||n||^2 - ||s_0 - s_1 + n||^2$ при раскрытии скобок преобразуется в 
\begin{equation}
\begin{split}
	||n||^2 - ||s_0 - s_1 + n||^2 = ||n||^2 - ||s_0 - s_1||^2 - 2\sum_{j=1}^{D}(s_{0j} - s_{1j}) n_j\\ - ||n||^2 = 
	 - ||s_0 - s_1||^2 - 2\sum_{j=1}^{D}(s_{0j} - s_{1j}) n_j.
\end{split}
\end{equation}
Произведем замену переменных $\Delta^2 =-2||s_0 - s_1||^2$ и $\epsilon =  -2\sum_{j=1}^{D}(s_{0j} - s_{1j}) n_j$, тогда 
\begin{equation}
	P_e(0) = Pr[\epsilon > \Delta^2].
\end{equation}
Стоит отметить, что $\epsilon$ является случайной гуассовской величиной, т.к. является линейной комбинацией разности сигналов и АБГШ. Принимая в расчет, что математическое ожидание АБГШ равно $0$, то $\bar{\epsilon} = 0$, а дисперсия $D[\epsilon] = 2N_0\Delta^2$.

Для дальнейших расчетов необходимо ввести Q-функцию, которая позволяет найти вероятность превышения некоторого порога $A$ гауссовской случайной величиной $x$ с параметрами $(m,\sigma^2)$:

\begin{equation}
	Pr[x > A] = Q(\frac{A - m}{\sigma}).
\end{equation}

Q-функция определяется следующей формулой:

\begin{equation}
	Q(x) = \int_{x}^{\infty}{\frac{1}{\sqrt{2\pi}}}e^{-z^2 / 2}dz
\end{equation} 

С учетом вышесказанного:
\begin{equation}
	P_e(0) = Q(\frac{\Delta^2}{\sqrt{\Delta^22N_0}}) = Q(\frac{\Delta}{\sqrt{2N_0}}).
\end{equation}

Так как $P_e(0)$ = $P_e(1)$, то
\begin{equation}
	P_e = Q(\frac{\Delta}{\sqrt{2N_0}}).
\end{equation}

Таким образом, вероятность ошибки дыоичных сигналов в канале с АБГШ зависит от величины евклидова расстояния между сигналами и интенсивности шума. При этом стоит учесть, что вид сигналов значения не имеет.
	%	\bibliographystyle{gost2008l}  %% стилевой файл для оформления по ГОСТу
	%	\bibliography{biblio}     %% имя библиографической базы (bib-файла) 
		\addcontentsline{toc}{subsection}{Список использованных источников}
\end{document}