% !TeX spellcheck = ru_RU
% !TeX encoding = UTF-8
\section{Теоретическая вероятность в двоичном канале с аддитивным белым гуассовским шумом}

Рассмотрим систему передачи двоичных сигналов $0$ и $1$. Вероятность ошибки в такой системе определяется по следующей формуле:

\begin{equation}
	P_e = \sum_{i=0}^{1}P_e(i)P_i = P_e(0)P_0 + P_e(1)P_1,
\end{equation}

где $P_e(i)$ -- условная вероятность ошибки при передаче $i$-го сигнала, $P_i$ -- вероятность передачи $i$-го сигнала. Рассмотрим простую систему с одинаковыми вероятностями передачи сигналов. Таким образом, можно рассчитать вероятность ошибки для одного символа, эта же вероятность будет являться вероятностью ошибки для всей системы. Вероятность ошибки будем рассчитывать по максимальному правдоподобию:
\begin{equation}
	\label{form:pe_1}
	P_e(0) = Pr[d^2(r,s_0) > d^2(r,s_1) \mid 0] = Pr[ || r - s_0||^2 > ||r - s_1||^2 \mid 0] ,
\end{equation}

где $r$ -- принятый сигнал, $s_i$ -- переданный $i$-ый сигнал. При этом нужно учесть, что $r = s_i + n$, где $n$ -- АБГШ, который описан в разделе \ref{sec:awgn}. Тогда формула (\ref{form:pe_1}) приобретает вид:
\begin{equation}
	P_e(0) = Pr[||n||^2 > ||s_0 - s_1 + n||^2] = Pr[||n||^2 - ||s_0 - s_1 + n||^2 > 0].
\end{equation}
Выражение $||n||^2 - ||s_0 - s_1 + n||^2$ при раскрытии скобок преобразуется в 
\begin{equation}
\begin{split}
	||n||^2 - ||s_0 - s_1 + n||^2 = ||n||^2 - ||s_0 - s_1||^2 - 2\sum_{j=1}^{D}(s_{0j} - s_{1j}) n_j\\ - ||n||^2 = 
	 - ||s_0 - s_1||^2 - 2\sum_{j=1}^{D}(s_{0j} - s_{1j}) n_j.
\end{split}
\end{equation}
Произведем замену переменных $\Delta^2 =-2||s_0 - s_1||^2$ и $\epsilon =  -2\sum_{j=1}^{D}(s_{0j} - s_{1j}) n_j$, тогда 
\begin{equation}
	P_e(0) = Pr[\epsilon > \Delta^2].
\end{equation}
Стоит отметить, что $\epsilon$ является случайной гуассовской величиной, т.к. является линейной комбинацией разности сигналов и АБГШ. Принимая в расчет, что математическое ожидание АБГШ равно $0$, то $\bar{\epsilon} = 0$, а дисперсия $D[\epsilon] = 2N_0\Delta^2$.

Для дальнейших расчетов необходимо ввести Q-функцию, которая позволяет найти вероятность превышения некоторого порога $A$ гауссовской случайной величиной $x$ с параметрами $(m,\sigma^2)$:

\begin{equation}
	Pr[x > A] = Q(\frac{A - m}{\sigma}).
\end{equation}

Q-функция определяется следующей формулой:

\begin{equation}
	Q(x) = \int_{x}^{\infty}{\frac{1}{\sqrt{2\pi}}}e^{-z^2 / 2}dz
\end{equation} 

С учетом вышесказанного:
\begin{equation}
	P_e(0) = Q(\frac{\Delta^2}{\sqrt{\Delta^22N_0}}) = Q(\frac{\Delta}{\sqrt{2N_0}}).
\end{equation}

Так как $P_e(0)$ = $P_e(1)$, то
\begin{equation}
	P_e = Q(\frac{\Delta}{\sqrt{2N_0}}).
\end{equation}

Таким образом, вероятность ошибки дыоичных сигналов в канале с АБГШ зависит от величины евклидова расстояния между сигналами и интенсивности шума. При этом стоит учесть, что вид сигналов значения не имеет.

\newpage