% !TeX spellcheck = ru_RU
% !TeX encoding = UTF-8
\subsection{Определение распределения Эрланга и его связь с экспоненциальным распределением}

Случайная величина $\eta$ имеет распределение Эрланга порядка $k>0$ с параметром $\lambda>0$, если

\begin{center}
	
	\begin{equation}
	\label {1.6}
	F_k(x) = P \{\eta<t\} = \begin{cases}1-e^{-\lambda t} \sum_{m=0}^{k-1}\frac{(\lambda t)^m}{m!},t>0,\\0,t\leq0.\end{cases}
	\end{equation} 
\end{center}


Распределение $F_k(t)$ имеет плотность


\begin{center}
	$ f_k(t)= F'_k(t) = \frac{\lambda(\lambda t)^{k-1}}{(k-1)!}e^{-\lambda t},t\geq0$,
\end{center}


что проверяется непосредственно дифференцированием функции \ref{1.6}, то есть справедливо равенство при $t>0$, $k>0$

\begin{center}
	\begin{equation}
	\label{1.7}
	\int_{0}^{1}\frac{\lambda(\lambda x)^{k-1}}{(k-1)!}e^{-\lambda t}dx = 1-e^{-\lambda t} \sum_{m=0}^{k-1}\frac{(\lambda t)^m}{m!}.
	\end{equation}
\end{center} 


Теперь установим связь между распределением Эрланга и экспоненциальным распределением. Нетрудно заметить, что распределение Эрланга первого порядка $(k=1)$ совпадает с экспоненциальным распределением.


Далее пусть задана последовательность $\{\xi_m, m>0\}$ независимых  в совокупности экспоненциально распределенных случайных величин с одним и тем же параметром $\lambda$. Обозначим $\eta_k = \sum_{m=1}^{k}\xi_m$ и покажем, что при любом $k>0$ случайная величина $\eta_k$ распределена по закону Эрланга порядка k c параметром $\lambda$, $P\{\eta_k<t\} = F_k(t)$.

Доказательство проведем, используя метод математической индукции. Как отмечалось выше при $k=1$ утверждение верно. Пусть оно верно при некотором $k>1$. Тогда получаем с учетом равенства \ref{1.7}

\begin{center}
	
	  $P\{\eta_{k+1}<t\} = \int_{0}^{1}P\{\xi_{k+1}<t-x\}f_k(x)dx = \int_{0}^{1}[1-e^{-\lambda(t-x)}]\frac{\lambda(\lambda x)^{k-1}}{(k-1)!}e^{-\lambda t}dx=$
\end{center}
\begin{center}
	  $= 1-e^{-\lambda t} \sum_{m=0}^{k-1}\frac{(\lambda t)^m}{m!} - e^{-\lambda
	   t}{\frac{(\lambda t)^k}{k!}} = F_{k+1}(t),$
\end{center}
  
  то есть утверждение верно при $k+1$. 
  
  Таким образом, доказано, что распределение Эрланга k-го порядка есть распределение суммы k независимых в совокупности, экспоненциально распределенных случайных величин с одним и тем же параметром $\lambda$.
  
\newpage

