% !TeX spellcheck = ru_RU
% !TeX encoding = UTF-8
\section{Нагрузка на сеть в Эрлангах}

Эрланг (обозначение Эрл) — безразмерная единица интенсивности нагрузки (чаще всего телефонной
нагрузки) или единица нагрузки, используемая для выражения величины нагрузки, требуемой для 
поддержания занятости одного устройства в течение определённого периода времени.

1 эрланг (1 Эрл) --- соответствует непрерывному использованию одного голосового канала в течение
1 часа. То есть если абонент проговорил с другим абонентом в течение одного часа, то на 
телекоммуникационном оборудовании была создана нагрузка в один эрланг.

Оценка телекоммуникационного трафика в эрлангах позволяет вычислить количество необходимых каналов
в конкретной зоне (области, базовой станции). Эрланг используется операторами связи для учёта 
пропускной способности при транзите трафика, так как телефонная нагрузка --- это случайная величина,
которая определяется количеством поступивших вызовов за единицу времени и временем обслуживания
абонента. Интенсивность нагрузки является произведением матожидания числа вызовов за единицу времени
на среднее время обслуживания вызова; эта интенсивность и измеряется в эрлангах. Важно отметить, что
введение рассматриваемой единицы, существенно, упростило расчёт нагрузки на сеть.

Единица названа в честь датского математика и инженера Агнера Крарупа Эрланга, который предложил
использовать математический анализ для учёта телефонной нагрузки. Агнер Эрланг проводил анализ
работы местной телефонной станции одной деревни, жители которой пытались установить соединение с
абонентами других населённых пунктов. В 1909 году им была опубликована работа «Теория вероятностей
и телефонные разговоры», в результате чего метод и стал популярным.