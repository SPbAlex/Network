% !TeX spellcheck = ru_RU
% !TeX encoding = UTF-8
\subsection{Сравнение систем: LoRa,«СТРИЖ» и Lonta Optima}
Результаты сравнения систем представлены в таблице \ref{tab:table1}. 
\begin{table}
	\centering
	\resizebox{\textwidth}{!}{%
		\begin{tabular}{|p{3cm}|p{3cm}|p{3cm}|p{3cm}|}\hline
			\begin{center}
				\textbf{Критерий сравнения системы }
			\end{center}&\begin{center}
			\textbf{LoRa}
		\end{center}&\begin{center}
		\textbf{«СТРИЖ»}
	\end{center}&\begin{center}
	\textbf{Lonta Optima}  
\end{center}
                                            \\ \hline
			Возможность передачи от приёмников к датчикам&\begin{center}
				есть
			\end{center}&\begin{center}
			нет
		\end{center}&\begin{center}
		нет
	\end{center}\\ \hline 
			Скорость передачи данных от датчиков к приемникам&\begin{center}
				50 Кбит/с
			\end{center}&\begin{center}
			 50 бит/с
		\end{center}&\begin{center}
		    50 бит/с\end{center}\\ \hline
			Дальность действия &до 5 км в условиях городской застройки и до 15 км в сельской местности&до 10 км в условиях городской застройки и до 50 км в сельской местности&до 10 км в условиях городской застройки и до 25 км в сельской местности\\ \hline 
			Частотный диапазон&868/915 МГц &2,4 ГГц, 868/915 МГц, 433 МГц, &433 МГц\\ \hline 
		\end{tabular}}
\end{table}
\newpage
\begin{table}
	\centering
	\resizebox{\textwidth}{!}{%
		\begin{tabular}{|p{3cm}|p{3cm}|p{3cm}|p{3cm}|}\hline
Открытые уровни&от канального уровня до прикладного уровня &все уровни являются закрытыми&все уровни являются закрытыми\\ \hline 
Закрытые уровни&физический уровень&все уровни являются закрытыми&все уровни являются закрытыми \\ \hline	\end{tabular}}
\caption{Результаты сравнения систем}.
\label{tab:table1}
\end{table}
\newpage