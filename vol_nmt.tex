\section{Мобильные сети первого поколения}
Мобильные сети первого поколения являются аналоговыми. Для передачи голоса в канале данных применялась частотная модуляция. Таким же способом осуществлялась передача управляющих команд в канале управления.

Наиболее известные стандарты первого поколения: Nordic Mobile Telephone (NMT) и Advanced Mobile Phone Service (AMPS).
NMT --- стандарт северных европейских стран, в том числе использовавшийся в России в качестве федерального стандарта. Предлагал два режима NMT-450 и NMT-900, диапазоны частот 450 МГц и 900 МГц соответственно.
AMPS --- широко распространенный стандарт Северной и Южной Америки. Диапазон частот 800 МГц. Также применялся в России, но как региональный стандарт.

Стандарты 1G обладали целым рядом недостатков, основным из которых является отсутствие шифрования. Любой абонент имел возможность перехватить данные, передаваемые в канале. К тому же, скорость передачи информации в 1G бала очень низкой (передача голоса 9.1 Kbit/s, передача данных 1.9 Kbit/s), что увеличивало стоимость разговора. Но, благодаря этим недостаткам, намечены векторы развития мобильных сетей.

\begin{table}[H]
    \centering
    \begin{tabular}{|c|c|c|}
        \hline
        Поколение & Канал данных & Канал управления\\
        \hline
        1G & Аналоговая модуляция & Аналоговая модуляция\\
        \hline
    \end{tabular}
\end{table}