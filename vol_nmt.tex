\section{Мобильные сети первого поколения}
Мобильные сети первого поколения являются аналоговыми. Для передачи голоса в канале данных применялась частотная модуляция. Таким же способом осуществлялась передача управляющих команд в канале управления.

Наиболее известные стандарты первого поколения: Nordic Mobile Telephone (NMT) и Advanced Mobile Phone Service (AMPS).
NMT --- стандарт северных европейских стран, в том числе использовавшийся в России в качестве федерального стандарта. Предлагал два режима NMT-450 и NMT-900, диапазоны частот 450 МГц и 900 МГц соответственно.
AMPS --- широко распространенный стандарт Северной и Южной Америки. Диапазон частот 800 МГц. Также применялся в России, но как региональный стандарт.

Стандарты 1G обладали целым рядом недостатков, основным из которых является отсутствие шифрования. Любой абонент имел возможность перехватить данные, передаваемые в канале. К тому же, скорость передачи информации в 1G бала очень низкой (передача голоса 9.1 Kbit/s, передача данных 1.9 Kbit/s), что увеличивало стоимость разговора. Но, благодаря этим недостаткам, намечены векторы развития мобильных сетей.

\begin{table}[h!]
	\centering
	\resizebox{\textwidth}{!}{%
		\begin{tabular}{|c|l|l|l|l|}
			\hline
			\textbf{Поколение}                                                          & \multicolumn{1}{c|}{\textbf{Канал данных}} & \multicolumn{1}{c|}{\textbf{Канал управления}}                          & \multicolumn{1}{c|}{\textbf{Разделение канала}} & \multicolumn{1}{c|}{\textbf{Защита канала}} \\ \hline
			\begin{tabular}[c]{@{}c@{}}1G\\ (AMPS, TACS, NMT-450, NMT-900)\end{tabular} & Аналоговый                                 & \begin{tabular}[c]{@{}l@{}}FM\\ \\ FDD\end{tabular}                     & FDMA                                            & Отсутствует                                 \\ \hline
			2G (GSM, CDMAOne)                                                           & Цифровой                                   & \begin{tabular}[c]{@{}l@{}}GMSK\\ \\ FDD\end{tabular}                   & TDMA/CDMA                                       & +                                           \\ \hline
			2,5G (GPRS)                                                                 & Цифровой                                   & \begin{tabular}[c]{@{}l@{}}GMSK\\ \\ FDD\end{tabular}                   & CDMA                                            & +                                           \\ \hline
			2,75(EDGE)                                                                  & Цифровой                                   & \begin{tabular}[c]{@{}l@{}}GMSK\\ \\ FDD\end{tabular}                   & CDMA                                            & +                                           \\ \hline
			3G(UMTS, CDMA2000)                                                          & Цифровой                                   & \begin{tabular}[c]{@{}l@{}}QPSK, 16QAM, 64QAM\\ \\ FDD+TDD\end{tabular} & W-CDMA                                          & +                                           \\ \hline
			LTE                                                                         & Цифровой                                   & \begin{tabular}[c]{@{}l@{}}QPSK, 16QAM, 64QAM\\ \\ FDD+TDD\end{tabular} & OFDMA                                           & +                                           \\ \hline
		\end{tabular}%
	}
	\caption{Сравнение поколений сетей}
	\label{net-table}
\end{table}
