% !TeX spellcheck = ru_RU
% !TeX encoding = UTF-8

\section{Децибел}
Децибел --- это единица измерения логарифмической величины, используемая для описания отношения двух величин физической природы. Децибел позволяет оценить степень усиления или затухания сигнала в системе и определяется следующим выражением:

\[ P_{db}=10\ln{(\dfrac{P}{P_{0}})} \]

Пусть $ P = 1 dB $, тогда:

\[\ln({\dfrac{P}{P_{0}}}) = 0,1\]

\[\dfrac{P}{P_{0}} = 10^{0,1}\]

\[\dfrac{P}{P_{0}} \approx 1,25893\]

Таким образом, увеличение энергетической величины на $ 1 dB $ означает её увеличение приблизительно в $ 1.25893 $ раза.
Величина $P_{0}$ является нулевым уровнем для $P$, и в случае равенства мощностей $P = P_{0}$, логарифм их отношения $lg(\dfrac{P_{1}}{P_{0}}) = 0$.

