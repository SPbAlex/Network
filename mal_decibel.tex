% !TeX spellcheck = ru_RU
% !TeX encoding = UTF-8

\subsection{Децибел}
Децибел --- это безразмерная единица, применяемая для измерения отношения некоторых величин. Является очень важной величиной для выражения усиления или затухания в системе в целом или в ее компонентах
Величина, выраженная в децибелах, численно равна десятичному логарифму безразмерного отношения физической величины к одноимённой физической величине, принимаемой за исходную, умноженному на десять:

\begin{equation}
A_{dB}=10 log(\frac{A}{A_{0}})
\end{equation}

где $A_{dB}$ — величина в децибелах, $A$ --- измеренная физическая величина, $A_{0}$ --- величина, принятая за базис.
Изначально дБ использовался для оценки отношения мощностей, и в каноническом, привычном смысле величина, выраженная в дБ, предполагает логарифм отношения двух мощностей и вычисляется по формуле:

\begin{equation}
$P_{dB}=10 log⁡(\frac{P_{1}}{P_{0}})$
\end{equation}

где x --- величина, измеряемая в дБ;$\frac{P_{1}}{P_{0}}$ --- отношение значений двух мощностей: измеряемой $P_{1}$ к так называемой опорной $P_{0}$, то есть базовой, взятой за нулевой уровень (имеется в виду нулевой уровень в единицах дБ, поскольку в случае равенства мощностей $P_{1} = P_{0}$ логарифм их отношения $lg(\frac{P_{1}}{P_{0}}) = 0$).

